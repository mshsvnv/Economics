\chapter{Метод целевых издержек для обеспечения конкурентоспособности}

В основу метода целевых издержек заложен следующий принцип: сначала определяют рыночную цену на данный вид продукции, затем устанавливают желаемый размер прибыли, а потом рассчитывают максимально допустимый размер себестоимости. 
Таким образом, допустимый размер себестоимости по методу \textbf{target costing} определяется следующим образом~\cite{target}: 
\begin{equation}
	\text{Цена - Прибыль = Себестоимость}.
\end{equation}

Рыночная цена в данном методе называется целевой ценой \textbf{(target price}), желательная разница между себестоимостью и продажной ценой называется целевой прибылью (\textbf{target profit}), а себестоимость, по которой изделие должно быть изготовлено~--- целевой себестоимостью (\textbf{target cost}).

Процесс установления целевой цены продукта предусматривает использование трехуровневого анализа качество продукта~--- набор его функциональных характеристик~--- цена продукта, где цена предполагается или задается как рынком в целом, так и непосредственными потребителями. Эта цена определяется с помощью маркетинговых исследований и фактически является ожидаемой рыночной ценой продукции. 
Целевая прибыль представляет величину прибыли, необходимую предприятию для развития и удовлетворения запросов собственников~\cite{target}.   

Схематично подход к ценообразованию с помощью целевых издержек представлен на рис.~\ref{img:img_01}. 
\includeimage
{img_01}
{f}
{H}
{.7\textwidth}
{Схема процесса формирования цены на основе использования  метода целевых издержек (target costing)~\cite{target}}

Управление издержками осуществляется не за счет чисто экономических приемов, а с использованием потенциала, содержащегося в оптимизации технических (конструкторских и технологических) решений, так как именно в данной области кроются огромные возможности по снижению и оптимизации стоимостных параметров технических систем. 
При этом изначальная фиксация цены и требуемых свойств позволит изделиям успешно конкурировать на рынке.

Метод \textbf{target costing} обладает такими преимуществами, как~\cite{target}:
\begin{itemize}
	\item планирование издержек;
	\item планирование прибыли;
	\item ориентация на спрос;
	\item ориентация на конкурентов;
	\item учет требований покупателей;
	\item дает возможность заранее учесть модификации цены (скидки, различия на внутреннем и внешнем рынках и т. д.); 
\end{itemize}

Метод \textbf{target costing} представляет собой эффективный инструмент для организаций, стремящихся к улучшению своих позиций на рынке через оптимизацию затрат и управление качеством. При правильной реализации он может значительно повысить гибкость, экономическую эффективность и общую конкурентоспособность компании~\cite{target}.
