\chapter{Конкурентоспособность}

Конкурентоспособность~--- способность бренда продавать продукты и услуги на рынке, который наполнен аналогичными товарами и услугами, а именно: привлекать клиентов, удерживать свою долю рынка и обеспечивать прибыльность. 
Предложения компаний по каким-либо причинам становятся более привлекательными по сравнению с аналогичными предложениями других.

\section{Определение конкурентоспособности}
Для организации высокая конкурентоспособность означает способность выигрывать в конкурентной борьбе, удовлетворять потребности и ожидания клиентов и, как следствие, достигать устойчивого и успешного развития на рынке.

Основные показатели, которыми оценивается конкурентоспособность компании, могут быть:
\begin{itemize}
	\item показатель, который отражает долю рынка, занимаемую организацией или продуктом. 
	Соответственно, большая доля рынка свидетельствует о более сильной конкурентоспособности;
	\item оценка темпов роста продаж позволяет судить о том, насколько успешно организация или продукт проникает на рынок и привлекает клиентов;
	\item Прибыльность организации оценивается уровнем доходности. 
	Более высокая прибыльность может свидетельствовать о преимуществе в конкурентной борьбе и более эффективном использовании ресурсов;
	\item Способность организации быстро реагировать на изменения во внешней среде и адаптироваться к новым требованиям рынка, в том числе открываться инновациям и внедрению новых идей.
\end{itemize}

\section{Оценка и определение уровня конкурентоспособности}
Для оценки уровня конкурентоспособности применяют:
\begin{itemize}
	\item  Исследование. 
	
	Анализ рынка, описание структуры, динамики и особенности конкуренции. Данные о доле рынка, темпах роста продаж и других показателях. Определение конкурентов.
	\item Сравнение. 
	
	Сопоставление показателей компании с показателями конкурентов.
	
	\item SWOT-анализ своих сильных и слабых сторон, а также возможностей и угроз. 
	
	Исследование клиентов и оценка репутации бренда необходимы для понимания их уровня удовлетворенности, предпочтений, отношения к бренду и конкурентам в сравнении.
	\item Анализ показателей производительности: себестоимость продукции, эффективность использования ресурсов, производительность труда.
\end{itemize}








