\chapter{Конкурентоспособность}

Конкурентоспособность~--- способность бренда продавать продукты и услуги на рынке, который наполнен аналогичными товарами и услугами, а именно: привлекать клиентов, удерживать свою долю рынка и обеспечивать прибыльность. 
Предложения компаний по каким-либо причинам становятся более привлекательными по сравнению с аналогичными предложениями других~\cite{enemy}.

\section{Определение конкурентоспособности}
Для организации высокая конкурентоспособность означает способность выигрывать в конкурентной борьбе, удовлетворять потребности и ожидания клиентов и, как следствие, достигать устойчивого и успешного развития на рынке.

Основные показатели, которыми оценивается конкурентоспособность компании, могут быть~\cite{enemy}:
\begin{itemize}
	\item показатель, который отражает долю рынка, занимаемую организацией или продуктом. 
	Соответственно, большая доля рынка свидетельствует о более сильной конкурентоспособности;
	\item оценка темпов роста продаж позволяет судить о том, насколько успешно организация или продукт проникает на рынок и привлекает клиентов;
	\item прибыльность организации оценивается уровнем доходности. 
	Более высокая прибыльность может свидетельствовать о преимуществе в конкурентной борьбе и более эффективном использовании ресурсов;
	\item способность организации быстро реагировать на изменения во внешней среде и адаптироваться к новым требованиям рынка, в том числе открываться инновациям и внедрению новых идей.
\end{itemize}

\section{Оценка и определение уровня конкурентоспособности}
Для оценки уровня конкурентоспособности применяют~\cite{enemy}
\begin{itemize}
	\item \textbf{Исследование}
	
	Этот шаг включает в себя анализ рынка и изучение особенностей конкуренции. 
	Данные о доле рынка, темпах роста продаж и других показателях помогут определить конкурентную среду, а также выявить сильные и слабые стороны компании относительно других участников рынка.
	\item \textbf{Сравнение}
	
	Сопоставление показателей компании с показателями конкурентов позволяет оценить ее положение на рынке, определить конкурентные преимущества или недостатки. 
	Это помогает выявить области, которые требуют дополнительного внимания для улучшения конкурентоспособности.
	\item \textbf{SWOT~-~анализ} своих сильных и слабых сторон, а также возможностей и угроз. 
	
	SWOT~-~анализ позволяет выявить сильные и слабые стороны компании, а также возможности и угрозы внешней среды.
	Исследование клиентов и оценка репутации бренда помогают понять уровень удовлетворенности клиентов, их предпочтения и отношение к компании и конкурентам.
	\item \textbf{Анализ показателей производительности}
	
	Этот анализ включает в себя оценку себестоимости продукции, эффективности использования ресурсов и производительности труда. 
	Понимание этих показателей позволяет оптимизировать процессы производства, снизить издержки и повысить конкурентоспособность компании на рынке.
\end{itemize}








