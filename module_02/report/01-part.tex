\chapter{Рынок}

Рынок~--- это система экономических отношений между продавцом и покупателем, которая позволяет совершать обмен необходимыми ресурсами. 
Понятие рынка неразрывно связано с товарами, так как продукт~--- это главная его единица. 
Благодаря разделению труда людям приходится обмениваться продукцией друг с другом, ведь рынок~--- это то, что возникает из-за ограниченности человеческих ресурсов. 
Когда один человек не может самостоятельно производить все, что ему нужно, ему приходится обменивать товары~--- это рождает рынок~\cite{market}.

\section{Признаки рынка}
Каждый вид рынка нацелен на разные продукты. 
Рынки эксплуатируют материальные объекты, умственные затраты и интеллектуальные труды. 
Они обладает неизменными признаками, по которым функционируют~\cite{market}:
\begin{itemize}
	\item \textbf{Нерегулируемый спрос} 
	
	Этот признак указывает на то, что потребители имеют свободу выбора в потреблении товаров и услуг. 
	Они могут решать, сколько именно им нужно приобрести и по какой цене.
	\item \textbf{Нерегулируемое предложение}
	
	Этот признак означает, что производители имеют свободу выбора в объеме производства товаров и услуг. 
	Они решают, сколько производить, основываясь на спросе и других факторах, таких как затраты производства и технические возможности.
	\item \textbf{Цена}
	 
	Этот признак указывает на то, что цена формируется на основе взаимодействия спроса и предложения. 
	При нерегулируемой цене рыночные силы определяют, на каком уровне установится цена, и она может меняться в зависимости от изменений в спросе и предложении.
\end{itemize}

\section{Функции рынка}
Рынок должен найти ответы на вопросы: что, как и для кого производить, поэтому он выполняет следующие функции~\cite{market}:
\begin{itemize}
	\item \textbf{Ценообразование} 
	
	Ценообразование осуществляется на основе взаимодействия спроса и предложения. 
	Цена отражает то, насколько потребители готовы платить за определенный продукт или услугу и насколько производители готовы предложить его.
	\item \textbf{Посредничество} 
	
	Рынок предоставляет механизм соединения потребителей и производителей. 
	Потребители могут выбирать из разнообразных предложений, а производители могут находить рынок для своей продукции.
	\item \textbf{Регулирование спроса и предложений}
	
	Рынок является механизмом, который помогает устанавливать баланс между спросом и предложением на товары и услуги.
	\item \textbf{Информирование}
	
	Рынок обеспечивает доступ к информации о товарах, продавцах, спросе на конкретные товары и общем состоянии экономики. 
	Это помогает принимать осознанные решения как потребителям, так и производителям.
	\item \textbf{Стимулирование} 
	
	Рыночная конкуренция стимулирует участников рынка постоянно улучшать качество товаров, разрабатывать новые технологии и повышать эффективность экономики в целом.
	\item \textbf{Очищение} 
	
	Рынок способствует освобождению экономики от ненужной деятельности и слабых участников, одновременно поощряя развитие перспективных секторов.
\end{itemize}
