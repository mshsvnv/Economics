\chapter{Рынок}

Рынок~--- это система экономических отношений между продавцом и покупателем, которая позволяет совершать обмен необходимыми ресурсами. 
Понятие рынка неразрывно связано с товарами, так как продукт~--- это главная его единица. 
Благодаря разделению труда людям приходится обмениваться продукцией друг с другом, ведь рынок~--- это то, что возникает из-за ограниченности человеческих ресурсов. 
Когда один человек не может самостоятельно производить все, что ему нужно, ему приходится обменивать товары~--- это рождает рынок.

\section{Признаки рынка}
Каждый вид рынка нацелен на разные продукты. 
Рынки эксплуатируют материальные объекты, умственные затраты и интеллектуальные труды. 
Они обладает неизменными признаками, по которым функционируют:
\begin{itemize}
	\item \textbf{Нерегулируемый спрос.} Потребитель решает, сколько ему нужно товаров и услуг.
	\item \textbf{Нерегулируемое предложение.} Производитель решает, сколько производить.
	\item \textbf{Цена}, которая не регулируется самостоятельно, а зависит от спроса и предложения.
\end{itemize}

\section{Функции рынка}
Рынок должен найти ответы на вопросы: что, как и для кого производить, поэтому он выполняет следующие функции:
\begin{itemize}
	\item \textbf{Ценообразование} 
	
	Цена устанавливается на основе законов рынка: спроса, предложения и возникающей конкуренции. Она отражает, насколько полезен данный продукт.
	\item \textbf{Посредничество} 
	
	Рынок соединяет потребителей и производителей, и у каждого есть свобода выбора.
	\item \textbf{Регулирование спроса и предложений. }
	
	Это устанавливает баланс товаров и услуг.
	\item \textbf{Информирование}
	
	Информация о товарах, продавце, размерах спроса на конкретные товары, состоянии экономики в целом.
	\item \textbf{Стимулирование} 
	
	Постоянно отвечая на спрос предложением, участники рынка повышают качество товаров, создают новые технологии и поддерживают эффективность экономики.
	\item \textbf{Очищение} 
	
	Экономика освобождается от ненужной деятельности, слабых единиц и поощряет развитие перспективных.
\end{itemize}
