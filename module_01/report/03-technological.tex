\chapter{Эластичность спроса и предложения}

Под эластичностью спроса и предложения понимают то, как эти показатели реагируют на изменение цены. 

На рисунке~\ref{img:img_03} представлены разные виды эластичности.
\includeimage
{img_03}
{f}
{H}
{.7\textwidth}
{Классификация эластичности~\cite{elast1}}

\section{Эластичность спроса}

Эластичность спроса показывает, как меняется поведение \textbf{покупателей}, когда продукция дорожает или дешевеет. 
Если понижение стоимости ведет к увеличению продаж, это значит, что спрос эластичный.
Когда потребители не реагируют на удешевление продукции, речь идет о неэластичном спросе.

Выделяют три измеримых вида эластичности спроса: по цене, по доходу и перекрестную~\cite{elast2}.

\subsection{По цене}
Помогает понять, как изменится спрос, если поднять или понизить стоимость продукта. 

Точечная эластичность спроса по цене рассчитывается по следующей формуле: $E_{D}^{P}=\frac{\Delta Q/Q}{\Delta P/P}$,
где верхний индекс $D$ \textit{(от англ. demand -- спрос)} означает, что это эластичность спроса, а нижний индекс $P$ \textit{(от англ. price -- цена)} говорит о том, что это эластичность спроса по цене.

Среди основных факторов, определяющих эластичность спроса по цене можно выделить следующие: 
\begin{itemize}
	\item временной фактор;
	\item доля расходов на товар в потребительском бюджете;
	\item степень насыщения рынка рассматриваемым товаром;
	\item важность товара для потребителя.
\end{itemize}

В зависимости от этих показателей различают:
\begin{itemize}
	\item \textbf{эластичный }спрос~--- если при изменении цены спрос на товар сильно меняется, такой спрос на товар считается эластичным;
	\item \textbf{неэластичный }спрос~--- если при изменении цены спрос на товар значительно не меняется, то спрос называют неэластичным;
	\item \textbf{абсолютно эластичный} спрос~--- когда есть только одна цена, при которой потребители будут покупать товар. 
	Изменение цены может привести к нулевому или неограниченному спросу;
	\item \textbf{абсолютно неэластичный} спрос~---вне зависимости от изменения цены спрос остается на прежнем уровне. 
\end{itemize}

На рисунке~\ref{img:elast_01} представлены графики эластичности спроса по цене.
\includeimage
{elast_01}
{f}
{H}
{.7\textwidth}
{Эластичность спроса по цене~\cite{elast1}}

К товарам с \textbf{эластичным спросом} можно отнести: драгоценности, автомобили, мебель, бытовая техника, деликатесы. 
К товарам с \textbf{неэластичным спросом}: лекарства, топливо для автомобиля, хлеб, соль. 
Эластичность спроса зависит от степени необходимости товара, наличия аналогов, доли расходов в личном или семейном бюджете (чем чаще человек покупает ту или иную продукцию, тем более предсказуемой будет его реакция на изменение цены).


\subsection{По доходу}
Показывает, как изменится спрос, если доходы потребителей вырастут или упадут. 
Помогает оценить перспективу конкретного сегмента рынка и скорректировать производство под возможности целевой аудитории.

Точечная эластичность спроса по цене рассчитывается по следующей формуле: $E_{I}^{D}=\frac{\Delta Q/Q}{\Delta I/I}$,
где верхний индекс $D$ \textit{(от англ. demand -- спрос)} означает, что это эластичность спроса, а нижний индекс $P$ \textit{(от англ. income -- доход)} говорит о том, что это эластичность спроса по доходу.

На основе анализа взаимосвязи между изменением дохода и спросом на разные товары можно построить график «доход~-~расходы» для отдельного блага (при предположении, что цена блага остается неизменной). 
Такие графики называют \textbf{кривыми Энгеля}. 
Это примерный вид кривых Энгеля для нормальных товаров (кривая $E_{1}$), предметов роскоши (кривая $E_{2}$) и низкокачественных товаров (кривая $E_{3}$). 
На оси абсцисс откладывается уровень дохода, а на оси ординат~--- расходы на потребление данного блага. 

\includeimage
{img_02}
{f}
{H}
{.4\textwidth}
{Прямые Энгеля~\cite{elast2}}

\subsection{Перекрестная эластичность}
Показывает, как изменится спрос на один товар, если поднять или понизить цену другого товара. 
Помогает определить, как стоимость заменяющих и дополняющих товаров влияет на желание покупателей приобрести продукт.

Точечная эластичность спроса по цене рассчитывается по следующей формуле: $E_{XY}^{D}=\frac{\Delta Q_{X}/Q_{X}}{\Delta P_{Y}/P_{Y}}$,
где верхний индекс $D$ \textit{(от англ. demand -- спроса)} означает, что это эластичность спроса, а нижний индекс $XY$ говорит о том, что это перекрёстная эластичность спроса, где под $X$ и $Y$ подразумеваются какие-то два товара. 
То есть перекрёстная эластичность спроса показывает степень изменения спроса на один товар  в ответ на изменение цены другого товара.

\section{Эластичность предложения}

Эластичность предложения, в свою очередь, демонстрирует, как \textbf{продавцы} реагируют на изменение цены: увеличивают или уменьшают объемы продукции на рынке. 

Эластичность предложения по цене зависит от:
\begin{itemize}
	\item срок хранения товара;
	\item издержек хранения, в том числе наличие свободных мест на складах;
	\item вероятности замены компонентов в производстве без потери качества;
	\item возможности быстро увеличить или уменьшить объемы производства в зависимости от роста или падения цен на продукцию.
\end{itemize}

Виды эластичности предложения:
\begin{itemize}
	\item \textbf{неэластичное} предложение~--- предложение существенно не меняется при изменении цены. Например, рынок рыбы. 
	Так как товар портится, его нужно продать. 
	К концу срока годности уже не важно, по какой цене;
	\item \textbf{эластичное} предложение~--- при изменении цены предложение существенно изменяется. 
	Свойственно товарам длительного хранения;
	\item \textbf{абсолютно неэластичное} предложение~--- как бы ни менялась цена, предложение будет постоянно на одном уровне;
	\item \textbf{абсолютно эластичное} предложение~--- существует только одна цена, по которой товар будет предлагаться на рынке. 
	Любое изменение цены приводит к полному отказу от производства либо к неограниченному увеличению предложения.
\end{itemize}

На рисунке~\ref{img:elast_02} представлены графики эластичности предложения по цене.
\includeimage
{elast_02}
{f}
{H}
{.7\textwidth}
{Эластичность предложения по цене~\cite{elast1}}

Пример \textbf{неэластичного предложения}~--- билет в кинотеатр. 
Даже если его цена за неделю вырастет на 100\%, продавец не сможет увеличить объем предложения: количество мест в зрительном зале ограничено.

Пример \textbf{эластичного предложения}~--- мороженое. 
Если в летние месяцы установилась сильная жара, спрос на продукцию увеличивается в разы даже при том условии, что цена растет. 
Производители могут быстро увеличить объемы производства и поставить на рынок дополнительные партии продукции. 
Или наоборот, снизить объемы производства, если лето выдалось холодным и мороженое практически никто не покупает.

Чем больше проходит времени с момента повышения цен или изменения спроса, тем более эластичными становятся спрос и предложение. 
У покупателей появляется возможность найти замену подорожавшей продукции или изменить свои привычки, а у продавцов~--- создать новые производственные мощности или избавиться от старых~\cite{elast1}.