\chapter{Эластичность спроса и предложения}

Под эластичностью спроса и предложения понимают то, как эти показатели реагируют на изменение цены. 

На рисунке~\ref{img:img_03} представлены разные виды эластичности.
\includeimage
{img_03}
{f}
{H}
{.7\textwidth}
{Классификация эластичности~\cite{elast1}}

\section{Эластичность спроса}

Эластичность спроса показывает, как меняется поведение \textbf{покупателей}, когда продукция дорожает или дешевеет. 
Если понижение стоимости ведет к увеличению продаж, это значит, что спрос эластичный.
Когда потребители не реагируют на удешевление продукции, речь идет о неэластичном спросе.

Выделяют три измеримых вида эластичности спроса: по цене, по доходу и перекрестную~\cite{elast2}.

\subsection{По цене}
Помогает понять, как изменится спрос, если поднять или понизить стоимость продукта. 

К тооварам с \textbf{эластичным спросом} можно отнести: драгоценности, автомобили, мебель, бытовая техника, деликатесы. 
К товарам с \textbf{неэластичным спросом}: лекарства, топливо для автомобиля, хлеб, соль. 
Эластичность спроса зависит от степени необходимости товара, наличия аналогов, доли расходов в личном или семейном бюджете (чем чаще человек покупает ту или иную продукцию, тем более предсказуемой будет его реакция на изменение цены).

\subsection{По доходу}
Показывает, как изменится спрос, если доходы потребителей вырастут или упадут. 
Помогает оценить перспективу конкретного сегмента рынка и скорректировать производство под возможности целевой аудитории.

На основе анализа взаимосвязи между изменением дохода и спросом на разные товары можно построить график «доход~-~расходы» для отдельного блага (при предположении, что цена блага остается неизменной). 
Такие графики называют \textbf{кривыми Энгеля}. 
Это примерный вид кривых Энгеля для нормальных товаров (кривая $E_{1}$), предметов роскоши (кривая $E_{2}$) и низкокачественных товаров (кривая $E_{3}$). 
На оси абсцисс откладывается уровень дохода, а на оси ординат~--- расходы на потребление данного блага. 

\includeimage
{img_02}
{f}
{H}
{.4\textwidth}
{Прямые Энгеля~\cite{elast2}}

\subsection{Перекрестная эластичность}
Показывает, как изменится спрос на один товар, если поднять или понизить цену другого товара. 
Помогает определить, как стоимость заменяющих и дополняющих товаров влияет на желание покупателей приобрести продукт.

\section{Эластичность предложения}

Эластичность предложения, в свою очередь, демонстрирует, как \textbf{продавцы} реагируют на изменение цены: увеличивают или уменьшают объемы продукции на рынке. 

Эластичность предложения по цене зависит от:
\begin{itemize}
	\item срок хранения товара;
	\item издержки хранения, в том числе наличие свободных мест на складах;
	\item вероятность замены компонентов в производстве без потери качества;
	\item возможность быстро увеличить или уменьшить объемы производства в зависимости от роста или падения цен на продукцию.
\end{itemize}

Пример \textbf{неэластичного предложения}~--- билет в кинотеатр. 
Даже если его цена за неделю вырастет на 100\%, продавец не сможет увеличить объем предложения: количество мест в зрительном зале ограничено.

Пример \textbf{эластичного предложения}~--- мороженое. 
Если в летние месяцы установилась сильная жара, спрос на продукцию увеличивается в разы даже при том условии, что цена растет. 
Производители могут быстро увеличить объемы производства и поставить на рынок дополнительные партии продукции. 
Или наоборот, снизить объемы производства, если лето выдалось холодным и мороженое практически никто не покупает.

Чем больше проходит времени с момента повышения цен или изменения спроса, тем более эластичными становятся спрос и предложение. 
У покупателей появляется возможность найти замену подорожавшей продукции или изменить свои привычки, а у продавцов~--- создать новые производственные мощности или избавиться от старых~\cite{elast1}.