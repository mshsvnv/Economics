\chapter{Закон спроса и предложения}

\textbf{Закон спроса и предложения}~--- это экономический принцип, описывающий взаимосвязь между ценой товара и его количеством, которое готовы приобрести покупатели (спрос) или продать продавцы (предложение).

\textbf{Закон спроса} гласит, что при увеличении цены товара количество спроса на него снижается, а при снижении цены~--- увеличивается. 
Иными словами, существует обратная зависимость между ценой товара и объёмом спроса на него.

\textbf{Закон предложения} утверждает, что при увеличении цены товара количество, которое производители готовы предложить на рынке, также увеличивается, а при снижении цены~--- уменьшается. То есть существует прямая зависимость между ценой товара и объёмом его предложения на рынке~\cite{law1}.

Эти два закона взаимосвязаны и определяют \textbf{рыночное равновесие}~--- точку, в которой объём спроса на товар равен объёму его предложения~\cite{law2}.

График~\ref{img:img_01} выражает одновременное поведение спроса и предложения отдельного товара и показывает, в какой точке две линии пересекутся. 

\includeimage
{img_01}
{f}
{H}
{.7\textwidth}
{График спроса и предложения~\cite{law2}}

В этой точке достигается равновесие. 
Координатами точки $Т$ являются равновесная цена $Р_{Т}$ и равновесный объем $Q_{T}$ (количество). 
Точка $Т$ характеризует равенство $Q_{Т} = Q_{S} = Q_{D}$, где $Q_{S}$~-- объем предложения, $Q_{D}$~-- объем спроса.

Точка равновесия показывает, что здесь спрос и предложение уравновешиваются. 
\textbf{Равновесная цена} означает, что товаров произведено столько, сколько требуется покупателям.

В экономической теории выделяются четыре правила спроса и предложения.
\begin{enumerate}
	\item увеличение спроса вызывает рост как равновесной цены, так и равновесного количества товара;
	\item уменьшение спроса приводит к падению равновесной цены и равновесного количества товара;
	\item рост предложения товара влечет за собой уменьшение равновесной цены и увеличение равновесного количества товара;
	\item cокращение предложения ведет к увеличению равновесной цены и уменьшению равновесного количества товара.
\end{enumerate}

Пользуясь этими правилами, можно найти равновесную точку в случае любых изменений спроса и предложения~\cite{law3}.