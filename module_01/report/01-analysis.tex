\chapter{Понятие спроса и предложения}

Под \textbf{спросом} понимается готовность, желание, намерение потребителей приобрести данное количество товара за определенный период времени по возможным ценам, подкрепленное денежной возможностью. 

Этот показатель может меняться в зависимости от:
\begin{itemize}
	\item цены;
	\item воздействия государства;
	\item сезона;
	\item уровня дохода населения;
	\item числа покупателей~\cite{law1}.
\end{itemize}

Когда цена на товар снижается, то объем спроса увеличивается. 
Но бывают и исключения: повышение цен влечет за собой увеличение спроса. 
У этих явлений есть своё название~-- \textbf{эффект Гиффена} и \textbf{парадокс Веблена}~\cite{law1}.
 
В первом случае речь идёт о жизненно важных бюджетных товарах, которые не имеют равноценных заменителей.
К ним относятся: рис, макароны, хлеб, чай.

Во втором~-- о престижных предметах роскоши. 
Для этих товаров спрос может расти даже при увеличении их цены~\cite{law2}. 

\textbf{Предложение}~--- это понятие, отражающее готовность, желание, намерение товаропроизводителя произвести (предложить) за определенный период времени определенное количество товара по той или иной цене.

Кроме того, на величину предложения влияют такие факторы, как:
\begin{itemize}
	\item себестоимость производства;
	\item конкуренция;
	\item прогнозы по цене и уровню потребности;
	\item государственное регулирование;
	\item налоги~\cite{law1}.
\end{itemize}

При возникновении на рынке избыточного предложении и падения цен, то события неизменно развиваются по одному из двух сценариев (или их комбинации):
\begin{enumerate}
	\item на рынке появляются новые покупатели;
	\item производители сокращают объемы производства.
\end{enumerate}

Первый вариант более благоприятен для рынка, но на практике случается редко. 
Чаще всего производителям приходится сокращать ресурсы, которые затрачиваются на выпуск данной категории товаров~\cite{law2}.

