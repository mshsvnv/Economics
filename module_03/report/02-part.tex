\chapter{Методы прогнозирования затрат}

Данные о затратах играют ключевую роль в управленческих решениях. Для управления и контроля за стоимостью недостаточно только фиксировать прошлые расходы. 
Руководитель проекта должен обратить внимание на будущие затраты. 
Даже если прошлые расходы были выгодными, они не могут быть изменены в будущем. 
Например, после завершения какой-то работы может быть обнаружено, что сделана бесполезная работа, но, к сожалению, ресурсы, потраченные на ее выполнение, не могут быть возвращены. 
Поэтому для контроля затрат необходимо прогнозировать будущее развитие проекта.

Процесс прогнозирования затрат включает три этапа:
\begin{enumerate}
	\item качественный анализ;
	\item количественный прогноз;
	\item проверка основных предположений количественного прогноза.
\end{enumerate}

\section{Качественный анализ}

Первый этап прогнозирования затрат такой же, как и при любом другом финансовом анализе: ознакомление с качественными проблемами, стоящими перед руководителем проекта. 
Поскольку прогноз основывается не только на достоверных, но и на определенных предположениях, необходимо изучить факторы, влияющие на процесс формирования затрат. 
Лучшее, с чего можно начать,~--- это оценить факторы, воздействующие на проект в целом. 
Важно помнить, что существует веская причина для того, чтобы уделить серьезное внимание вопросам качественного анализа: менеджеру, занимающемуся прогнозированием, крайне важно установить, какие именно предположения окажут решающее влияние на его прогноз.

После проведения качественного анализа менеджер способен сформулировать свое мнение о предсказуемости хода проекта. Оценивая особенности управления проектом, мощности, размеры, график работ, структуру издержек, менеджер определяет критические параметры, существенно влияющие на проект. 
Эти параметры и будут теми основными предположениями, которые лягут в основу прогнозирования затрат.

\section{Количественный прогноз}

На этой стадии анализа очень важны временные рамки. Наиболее важные для развития проекта факторы, выявленные на предыдущем этапе, помогают определить направление прогноза и установить момент будущего, когда прогноз потеряет свое значение.

Количественный прогноз включает в себя \textit{текущую оценку затрат} и \textit{прогноз итоговых затрат.}

\subsection{Текущая оценка затрат}
Эта оценка учитывает любые изменения, произошедшие с момента осуществления проекта. 
Текущая информация о степени завершенности работ может быть получена на основе субъективных заключений менеджеров о проценте завершенности работы. 
Естественно, этот метод оценки пристрастен из-за оптимизма, пессимизма или ошибочных наблюдений.

\section{Прогноз итоговых затрат}
Данный прогноз с учетом действительного выполнения проекта и фактически произведенных расходов может быть получена несколькими методами:

\begin{itemize}
	\item \textbf{Метод простой линейной экстраполяции} стоимости работы.
	
	Используя линейную экстраполяцию стоимостей, прогнозную итоговую стоимость $C_{пр}$ можно получить по формуле:
	\begin{equation} 
		C_{\text{пр}} = \frac{C_t}{V_t}, 
	\end{equation}
	где $C_{t}$~--- фактические затраты на момент времени $t$, $V_{t}$~--- доля работ, выполненная за время $t$.
	
	Если стоимость работы распределена по элементам (статьям) затрат, то итоговая стоимость является суммой прогнозных стоимостей по каждому элементу.
	
	Оценка итоговых затрат может быть изменена при прогнозировании, если планируется изменить (снизить или повысить) стоимость оставшегося объема работ за единицу времени. 
	В этом случае оценочное уравнение прогнозной итоговой стоимости работы имеет вид:
	\begin{equation}
		C_{\text{пр}} = С_{t} + (V - V_{t}) \cdot C,
	\end{equation}
	где $C_{\text{пр}}$~--- прогнозная итоговая стоимость, $С_{t}$~--- фактическая (реальная) стоимость на дату $t$, $(V - V_{t})$~--- оставшийся объем работ, $C$~--- ожидаемая стоимость оставшейся работы на единицу объема.
	
	\item \textbf{Метод стоимостной пропорции}
	
	Он оценивает фактические затраты на определенную дату и используется для вычисления оценочного процента выполнения работы. 
	Этот метод не обеспечивает независимой информации о действительном проценте завершенности. 
	Поэтому менеджеры должны использовать оценочные стоимости о завершенности работы по методу стоимостной пропорции очень осторожно.
	\includeimage
	{img_01}
	{f}
	{H}
	{.7\textwidth}
	{Соотношение завершенности работы и расходов по ней}
	
	В точке А расходы выше, чем предполагалось. 
	Эта точка отражает 40\% завершенности работы и 50\% расхода бюджета. 
	Так как по плану предполагалось потратить только 40\% бюджета для завершения 40\% работы, то 40-50 =-10\% составляет перерасход стоимости. 
	Если превышение стоимости продолжает расти (прогнозируемые расходы), то итоговая стоимость выполнения работы всегда будет выше запланированной.
	
	Таким образом, прогнозирование затрат позволяет подтвердить нехватку или излишек средств еще до их возникновения и дает достаточную возможность своевременно предпринять корректирующие воздействия.
	
	\item \textbf{Ресурсный метод} 
	
	Данный метод прогнозирования затрат представляет собой калькулирование в прогнозируемых ценах и тарифах элементов затрат (ресурсов), необходимых для реализации проекта. Калькулирование предстоящих издержек реализации проекта ведется на основе выраженных в натуральных измерителях потребностей в ресурсах.
\end{itemize}

