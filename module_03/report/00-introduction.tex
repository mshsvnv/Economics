\chapter*{ВВЕДЕНИЕ}
\addcontentsline{toc}{chapter}{ВВЕДЕНИЕ}

Деятельность любого коммерческого предприятия направлена на получение прибыли. 
Полученная прибыль, в свою очередь, зависит не только от поступающих сумм выручки, но и от возникающих в процессе хозяйственной деятельности затрат и расходов. 
Чем меньше расходная часть предприятия, тем выше прибыль и финансовый результат от хозяйственной деятельности.

Для минимизации расходной части затраты должны контролироваться. Для выполнения этой задачи необходимо, как минимум, корректно понимать, где, как и почему они возникают. 
И есть ли ресурсы на их оптимизацию.