\chapter{Затраты проекта}

После того как план проекта и график будет готов, можно начать подготовку бюджета. 
Раньше это сделать невозможно, потому что все проектные решения влияют в одном или другом  виде на затраты, необходимые для реализации проекта~\cite{project2}.

\section{Типы затрат проекта}
Стоимость проекта можно разделить по-разному. Наиболее традиционно разделение на~\cite{project2}:
\begin{itemize}
	\item прямые и косвенные затраты;
	\item затраты на проведение проекта, и затраты новой внедряемой системы. 
\end{itemize}

\section{Прямые и косвенные затраты}
\textbf{Прямые затраты}~--- это затраты, непосредственно относимые к производственному процессу при изготовлении продукции, выполнении работ или услуг. 
В частности, в производственной деятельности~--- это стоимость материалов, амортизация производственной техники, зарплата и взносы с неё производственных рабочих. 
В торговле к прямым затратам относятся стоимость продаваемых товаров, их транспортировки до покупателя, страховые расходы, пошлины.

\textbf{Косвенные расходы}~--- это затраты, не зафиксированные в качестве прямых в учётной политике и не отнесённые к внереализационным.
Косвенные расходы включают в себя, например, зарплату бухгалтерам или юристам, оплату коммунальных услуг и др.

Прямые и косвенные затраты являются критически важными аспектами финансового учета, позволяя предприятиям эффективно управлять ресурсами и принимать обоснованные решения. 
Их правильное понимание и учет помогают организациям максимально увеличивать прибыль и снижать издержки, создавая устойчивый бизнес.

\section{Затраты на проведение проекта и систему}

\subsection{Затраты на проведение проекта}

Затраты на проведение проекта охватывают все необходимые ресурсные и финансовые затраты, возникающие на всех этапах реализации. Это включает в себя~\cite{project3}:
\begin{itemize}
	\item Зарплаты сотрудников 
	
	Основная часть затрат, связанных с реализацией проекта, относится к заработной плате ключевых участников команды. 
	Важно учитывать не только фиксированные зарплаты, но и дополнительные выплаты, такие как премии и бонусы за выполнение ключевых этапов проекта.
	\item Материалы и ресурсы
	
	Дополнительно могут возникать затраты на хозяйственные нужды проекта, включая канцелярские товары, программное обеспечение и другое оборудование, необходимое для выполнения задач.
	\item Обучение и развитие
	
	В случае внедрения новых технологий могут потребоваться затраты на обучение сотрудников, чтобы они могли эффективно работать с новыми системами и процессами.
	\item Аудит и консультационные услуги
	
	Иногда для обеспечения успешности проекта нанимаются внешние консультанты или аудиторы, что также влечет за собой дополнительные расходы.
\end{itemize}


\subsection{Затраты на систему}
	
Затраты на систему характеризуют все финансовые обязательства, связанные с внедрением и поддержкой новой информационной или организационной системы. Эти затраты можно разделить на два основных типа~\cite{project3}:
\begin{enumerate}
	\item Одноразовые
	
	Эти затраты возникают в ходе реализации проекта и включают:
	\begin{itemize}
		\item \textbf{Стоимость разработки:} затраты на создание новой системы, включая программное обеспечение, оборудование, и интеграцию различных технологий.
		\item \textbf{Затраты на лицензии и сертификации:} приобретение необходимых лицензий на программное обеспечение и получение сертификатов для соответствия стандартам и требованиям.
		\item \textbf{Системное тестирование и отладка:} затраты, связанные с тестированием системы после её разработки, чтобы обеспечить её корректное функционирование.
	\end{itemize}
	\item Постоянные
	
	Постоянные затраты относятся к рабочему циклу эксплуатации созданной системы и могут включать:
	\begin{itemize}
		\item \textbf{Обслуживание и поддержка:} затраты на техническую поддержку системы, включая зарплаты специалистов, занимающихся её мониторингом и исправлением ошибок.
		\item \textbf{Обновления и модернизация:} необходимость периодического обновления программного обеспечения и аппаратуры для поддержки актуальности и эффективной работы системы.
		\item \textbf{Затраты на безопасность:} обеспечение защиты данных и информационной безопасности системы требует дополнительных инвестиций на систему защиты и обучение сотрудников.
	\end{itemize}
\end{enumerate}

Затраты на проведение проекта и системы требуют внимательного учета и планирования. 
Понимание структуры затрат позволяет организациям эффективно управлять ресурсами и контролировать бюджет, что в конечном итоге способствует достижению успешных результатов проекта и повышению его прибыльности~\cite{project3}. 

