\chapter{Затраты проекта}

После того как план проекта и график будет готов, можно начать подготовку бюджета. 
Раньше это сделать невозможно, потому что все проектные решения влияют в одном или другом  виде на затраты, необходимые для реализации проекта.

\section{Типы затрат проекта}
Стоимость проекта можно разделить по-разному. Наиболее традиционно разделение на:
\begin{itemize}
	\item прямые и косвенные затраты;
	\item затраты на проведение проекта, и затраты новой внедряемой системы. 
\end{itemize}

\subsection{Прямые и косвенные затраты}
\textbf{Прямые затраты}~--- это затраты, непосредственно относимые к производственному процессу при изготовлении продукции, выполнении работ или услуг. 
В частности, в производственной деятельности~--- это стоимость материалов, амортизация производственной техники, зарплата и взносы с неё производственных рабочих. 
В торговле к прямым затратам относятся стоимость продаваемых товаров, их транспортировки до покупателя, страховые расходы, пошлины.

\textbf{Косвенные расходы}~--- это затраты, не зафиксированные в качестве прямых в учётной политике и не отнесённые к внереализационным.
Косвенные расходы включают в себя, например, зарплату бухгалтерам или юристам, оплату коммунальных услуг и др.

\subsection{Затраты на проведение проекта и систему}
\textbf{Затраты на проведение проекта} это, как правило, возникающие в ходе проекта разовые затраты на зарплаты сотрудников.

\textbf{Затраты на систему}~--- это затраты, возникающие при внедрении новой системы. 
Их можно разделить на:
\begin{itemize}
	\item Одноразовые
	
	Это все затраты, которые возникают в ходе проекта при создании системы и не являются проектными затратами.
	\item Постоянные
	
	Это все затраты, которые возникают во время эксплуатации системы, созданной в ходе проекта.
\end{itemize}

