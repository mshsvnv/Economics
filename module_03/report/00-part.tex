\chapter{Проект}

Проект~--- это временное предприятие, направленное на создание уникального продукта, услуги или результата. 
В отличие от повторяющихся операций, которые также могут создавать ценность для компании, проекты уникальны по своей природе. 
Рассмотрим более подробно ключевые аспекты, связанные с проектами.

В этом определении важно следующее~\cite{project1}:
\begin{itemize}
	\item Проекты создают уникальные результаты. 
	Проекты направлены на создание чего-то нового, будь то продукт, услуга или даже исследование. Например, разработка нового программного обеспечения или создание маркетинговой кампании требует уникального подхода и ресурсов. 
	Эта уникальность возникает из-за потребности удовлетворения специфических требований заказчика или рынка, отличающихся от стандартной практики компании. 
	Это также подразумевает использование инновационных технологий и методов, способствующих получению новых знаний и опыта.
	\item Проекты ограничены по времени. 
	Временные рамки могут быть различными: от нескольких дней до нескольких лет. 
	Четкое определение временных границ помогает команде сосредоточиться на ключевых задачах и эффективно распределять ресурсы. 
	Кроме того, наличие дедлайнов способствует контролю за выполнением и уровнем качества.
\end{itemize}

Любой проект имеет начало, середину и конец. 
Это и есть жизненный цикл~--- если обобщать. 

\textbf{Жизненный цикл}~--- это последовательность этапов, через которые проходит любой проект. 
После окончания цикла появляется конкретный результат. 
Реализация проекта с помощью жизненного цикла помогает организованно достигать поставленных целей и структурировать весь процесс~\cite{project1}.
	
У каждого проекта есть пять основных этапов: 
\begin{enumerate}
	\item инициация;
	\item планирование;
	\item реализация;
	\item мониторинг и контроль;
	\item завершение.
\end{enumerate}

\section{Мониторинг и контроль}
Четвертый этап по времени совпадает с третьим. Текущий мониторинг включает такие процессы:
\begin{itemize}
	\item рациональное расходование времени; 
	\item рациональное расходование ресурсов: людей, времени и бюджета.
\end{itemize}

На этом этапе нужна полная прозрачность процессов, чтобы вовремя устранять недочёты, корректировать план, общаться внутри коллектива. Часто в этом помогают дополнительные совещания.