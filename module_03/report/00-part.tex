\chapter{Проект}

Проект~--- это временное предприятие, направленное на создание уникального продукта, услуги или результата. 

В этом определении важно следующее:
\begin{itemize}
	\item проекты создают уникальные результаты. 
	Они могут быть и материальными, и нематериальными. 
	Такие результаты отличаются от того, что компания делает каждый день и уже превратила в процесс;
	\item проекты ограничены по времени. 
	У них есть чётко очерченные начало и конец;
\end{itemize}

Любой проект имеет начало, середину и конец. 
Это и есть жизненный цикл~--- если обобщать. А вот конкретно:

	\textbf{Жизненный цикл}~--- это последовательность этапов, через которые проходит любой проект. 
	После окончания цикла появляется конкретный результат. Реализация проекта с помощью жизненного цикла помогает организованно достигать поставленных целей и структурировать весь процесс.
	
По методологии Института управления проектами у каждого проекта есть пять основных этапов: 
\begin{enumerate}
	\item инициация;
	\item планирование;
	\item реализация;
	\item мониторинг и контроль;
	\item завершение.
\end{enumerate}

\section{Мониторинг и контроль}
Четвертый этап по времени совпадает с третьим. Текущий мониторинг включает такие процессы:
\begin{itemize}
	\item Рациональное расходование времени. 
	\item Рациональное расходование ресурсов: людей, времени и бюджета.
\end{itemize}
На этом этапе нужна полная прозрачность процессов, чтобы вовремя устранять недочёты, корректировать план, общаться внутри коллектива. Часто в этом помогают дополнительные совещания.